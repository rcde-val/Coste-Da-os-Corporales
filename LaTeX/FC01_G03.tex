\documentclass[11pt]{article}
\usepackage[spanish]{babel}  % Idioma español
\usepackage[utf8]{inputenc}
\usepackage{setspace}
    \singlespacing  % Espaciado sencillo
    \usepackage[a4paper,margin=2.5cm]{geometry}  % Márgenes de 2.5 cm
\usepackage{helvet}  % Usa Helvética, que es muy similar a Arial
    \renewcommand{\familydefault}{\sfdefault}  % Aplica fuente sans-serif (Helvética)
\usepackage[colorlinks=true,linkcolor=black]{hyperref}  % Hipervínculos en el índice
\usepackage{hyperref} % Hipervínculos
\usepackage{graphicx} % Imágenes
\usepackage{subcaption} % Imágenes
\graphicspath{{FC01_G03_Imagenes/}}
\usepackage{tabularx} % Tablas
\usepackage{booktabs} % Tablas
\usepackage{multirow} % Tablas
\usepackage{float} % Tablas
\usepackage{caption} % Tablas
\usepackage{bookmark}
\usepackage{threeparttable} % Notas al pie de tablas

\title{\textbf{Costes esperados por daños corporales en seguro de automóviles e influencia en reservas\\[2em]}}
\author{\underline{Grupo 3} \\ Víctor Alonso Lara \\ David López Avakian \\ Sergio Obando Henao \\ Víctor Manuel Pérez \\ Miquel Trullols Salat \\[6em]}
\date{19 de Diciembre de 2025}

\begin{document}

% Portada
\begin{titlepage}
    \maketitle
    \thispagestyle{empty}
\end{titlepage}
\newpage

% Índice
\tableofcontents
\newpage

% Sección 1: Objetivo
% Fecha de entrega 1: 31/10/25 (Máximo 4 páginas en total)
\section{Objetivo}

XEste estudio tiene como objetivo aplicar los conocimientos en modelos estadísticos y cuantificación de riesgos al análisis de un caso práctico en el ámbito del seguro de automóviles. En particular, se centra en el estudio de los costes esperados por daños corporales y su influencia en el cálculo de reservas técnicas. Se trata de un tema de gran relevancia, especialmente en el mercado español, donde la compensación por daños corporales representa más del 60\% del coste total en el seguro de responsabilidad civil del automóvil \cite{santolino_prieto_metodos_2011}. \\

Para su desarrollo, el trabajo se estructura en dos fases:

- Revisión de la literatura existente, con especial atención a los modelos utilizados en la estimación de costes y reservas.

- Modelización del número y la cuantía de los siniestros, mediante técnicas estadísticas aplicadas a datos reales. \\

Antes de abordar el análisis, se presentarán dos conceptos fundamentales que permiten contextualizar el problema y establecer las bases teóricas del estudio:

- En España, el \textbf{seguro obligatorio de automóviles} está regulado por el Real Decreto Legislativo 8/2004. La ley establece que todo propietario de un vehículo a motor con estacionamiento habitual en España debe contratar y mantener un seguro que cubra la responsabilidad civil por los daños causados a personas o bienes durante la circulación del vehículo \cite{noauthor_boe--2004-18911_nodate}.

- Los \textbf{daños corporales} son las lesiones físicas o psíquicas sufridas por una persona en un accidente de circulación, incluyendo lesiones temporales, secuelas permanentes y fallecimiento. Su valoración e indemnización se regulan en la Ley 35/2015 mediante el Baremo oficial \cite{jefatura_del_estado_ley_2015}. 
    
% Sección 2: Estado de la cuestión: Revisión de la literatura existente
% Fecha de entrega 1: 31/10/25 (Máximo 4 páginas en total)
% Comentarios: 
% - La literatura encontrada y usada debe ser agrupada por ítems o ideas fundamentales. Se trata de agrupar de forma conjunta todos aquellos trabajos que se centren en aproximaciones similares, diferenciándolos del resto.
% - Deben quedar claramente diferenciados los artículos que desde el punto de vista del alumno se asocian a la asignatura de Modelos estadísticos y los que se asocian a la asignatura de Cuantificación de riesgos, indicando como decíamos en el punto anterior, sus contribuciones fundamentales. La asociación a cada asignatura debe ser fácilmente localizable en el texto.
% - Debe acompañarse la entrega con el listado de referencias bibliográficas usadas (también se explicará en clase como hacerlo) – entrega de librería en formato de Excel.
\section{Estado de la cuestión: Revisión de la literatura existente}

\subsection{Número de siniestros}

La frecuencia de siniestros es un elemento fundamental en la estimación de los costes esperados por daños corporales en el seguro del automóvil, ya que una predicción más precisa del número de reclamaciones permite ajustar con mayor exactitud las reservas técnicas y reducir la incertidumbre asociada a los siniestros futuros. En los últimos años, distintos estudios han buscado mejorar la modelización de esta variable mediante enfoques
capaces de reflejar la complejidad y las dependencias que caracterizan los datos de
siniestralidad. \\

Los siniestros con daños corporales constituyen un subconjunto específico dentro del conjunto total de siniestros en el seguro de automóviles, dado que no todos los siniestros implican necesariamente lesiones físicas. \\

En este contexto, Álvarez Jareño y Muñiz Rodríguez analizan la idoneidad de las distribuciones clásicas para modelizar el número de siniestros en carteras de seguros de responsabilidad civil de automóviles. A partir del estudio de 15 carteras, los autores identifican diversas anomalías muestrales recurrentes que cuestionan la validez de la distribución de Poisson como modelo base. Entre estas anomalías destacan el contagio, la sobre-dispersión (varianza superior a la media), el inflado de ceros (frecuencia excesiva de asegurados sin siniestros), el desinflado de unos (subestimación de asegurados con un único siniestro) y la presencia de colas más pesadas (subestimación de conductores con múltiples siniestros). Estas irregularidades evidencian que el
supuesto de independencia entre eventos y la igualdad entre media y varianza del modelo de Poisson no se cumplen en la práctica \cite{alvarez_jareno_reparametrizacion_2010}. \\

Para abordar estas limitaciones, los autores proponen la reparametrización de distribuciones alternativas que ofrecen un mejor ajuste a los datos observados: la distribución binomial negativa, la distribución Polya-Aeppli, la distribución Poisson
Inversa Gaussiana y la distribución Poisson Pascal Generalizada. Estas distribuciones permiten capturar la presencia de colas pesadas, proporcionando así una base más robusta para la modelización de la frecuencia siniestral. \\

Otros autores como Pechon, Trufin y Denuit (2018) analizan la frecuencia de siniestros en el seguro obligatorio de responsabilidad civil automotriz tomando al hogar como unidad de riesgo. A diferencia de los modelos que tratan cada póliza de manera independiente, los autores incorporan efectos
aleatorios correlacionados a través de mezclas Poisson–LogNormal y Poisson–Gamma, con el propósito de capturar la dependencia entre los miembros de un mismo hogar y la heterogeneidad no observada. Los resultados muestran una correlación significativa entre
las siniestralidades de los cónyuges, cercana al 40\%, lo que confirma la existencia de una propensión común al riesgo. Este enfoque permite afinar las estimaciones de frecuencia y,
en consecuencia, mejorar la valoración de los daños corporales y la suficiencia de las provisiones técnicas \cite{pechon_multivariate_2018}. \\

Otro enfoque a tener en cuenta es el de \cite{tzougas_bivariate_2023}. En su artículo, han desarrollado una clase de modelos de regresión Poisson bivariados mixtos con dispersión variable, orientados a modelizar de forma conjunta la frecuencia de
reclamaciones por daños corporales y la frecuencia de reclamaciones por daños materiales en el seguro de responsabilidad civil de automóviles. Estos modelos incorporan distribuciones de mezcla para capturar la variabilidad no explicada y permiten
analizar simultáneamente las dos variables correlacionadas. Además, son capaces de reflejar tanto la sobre dispersión como la correlación positiva entre ambas frecuencias, lo que representa un avance significativo en la modelización multivariada del riesgo.\\

La literatura revisada evidencia que comprender la frecuencia de siniestros no solo mejora la precisión en la estimación de los daños corporales, sino que también constituye un componente esencial en la gestión del riesgo y en la sostenibilidad del sistema asegurador.

\subsection{Cuantía del siniestro}

Los siniestros con daños corporales en el seguro de automóviles se caracterizan por una alta variabilidad en sus costes. En España, durante 2005, la mayoría de estos siniestros costaron menos de 1.500 €, pero un 0,5\% superaron los 300.000 €, y algunos casos graves, como lesiones tetrapléjicas, pueden superar el millón de euros \cite{SantolinoMiguel2007Urmd}. \\

Como primera aproximación, se realiza un análisis descriptivo de los factores que determinan el coste por daños corporales. Por ejemplo, Marter y Weisberg (1991) clasifican los siniestros de tráfico en cuatro categorías según el tipo de lesión sufrida por la víctima —esguince, fractura, contusión y herido grave— y, para cada una, comparan elementos como el coste médico total, el coste sin hospitalización, el proveedor de asistencia, la frecuencia de visitas y el período de curación [Santolino Prieto, 2011]. Santolino también subraya la relevancia del análisis descriptivo como punto de partida en su estudio sobre indemnizaciones por daños corporales en seguros de auto fijadas judicialmente en Cataluña y Aragón durante el período 2001-2003 \cite{santolino_prieto_metodos_2011}.\\

A partir del análisis descriptivo, la literatura propone modelos para estimar el coste de los siniestros. Weisberg y Derrig \cite{santolino_prieto_metodos_2011} plantean el uso del modelo Tobit, adecuado para datos censurados, donde la indemnización no puede ser inferior a cero ni superar los límites legales o de póliza. La variable dependiente es la indemnización (continua y censurada), mientras que las explicativas incluyen factores dicotómicos —contratación de abogado, lesión grave, fractura, indicios de exageración— y cuantitativos —porcentaje de culpa, coste médico total, semanas de incapacidad—. Este enfoque permite estimar el impacto de factores médicos y legales sobre el logaritmo de la indemnización esperada, ajustando por censura. Los resultados muestran que la contratación de un abogado, la clasificación de la lesión como grave y la presencia de fracturas incrementan la indemnización, mientras que los indicios de exageración la reducen.\\

El modelo logit ordenado es una herramienta estadística adecuada para analizar variables categóricas jerárquicas, especialmente cuando las categorías tienen un orden natural. En el contexto de siniestros, este modelo permite clasificar la severidad de los eventos en distintos niveles como leve, moderado y grave, y evaluar cómo diferentes factores influyen en la probabilidad de que un siniestro pertenezca a una categoría de mayor severidad.
El objetivo principal de este modelo es identificar los factores que incrementan la probabilidad de que un siniestro se clasifique en niveles superiores de pérdida, lo que resulta fundamental para la gestión del riesgo. Para abordar limitaciones del modelo logit ordenado clásico y capturar mejor la complejidad de los datos, se pueden considerar varias extensiones: ordenado mixto, ordenado heterocedástico y  multinomiales \cite{santolino_prieto_metodos_2011}.\\

Santolino propone un modelo econométrico log-lineal para explicar el logaritmo de la indemnización total, incorporando variables como edad, tipo de lesión, tipo de vehículo y sexo del lesionado. Un hallazgo relevante es que ni el tipo de vehículo ni la edad del conductor resultan significativos al 10\% de nivel de confianza. Además, se observa que las mujeres reciben indemnizaciones mayores que los hombres y que, cuando el perito necesita más de una visita a la víctima, la cuantía indemnizatoria tiende a incrementarse \cite{santolino_prieto_metodos_2011}. \\

La predicción del coste de indemnización es un aspecto crítico para las compañías aseguradoras, ya que determina la capacidad de la entidad para cumplir con sus obligaciones futuras. Las aseguradoras deben disponer de reservas suficientes que garanticen la estabilidad financiera y la solvencia del ramo. Este desafío se intensifica en los siniestros corporales cuya indemnización se reclama por vía judicial, dado que la resolución suele demorarse durante meses o incluso años. En consecuencia, estos expedientes permanecen abiertos en la contabilidad de la compañía, lo que obliga a realizar provisiones adecuadas para cubrir el coste esperado. \cite{SantolinoMiguel2007Urmd}. \\

Por último, se recomienda a las aseguradoras prestar especial atención a los siniestros que superan el percentil 90-95\%, ya que representan casos atípicos con costes significativamente elevados. Estos expedientes requieren una evaluación más exhaustiva para verificar la consistencia de los gastos médicos reclamados y detectar posibles exageraciones o prácticas fraudulentas. Un análisis detallado en esta franja no solo contribuye a reducir el riesgo de sobre indemnización, sino que también permite optimizar la asignación de reservas \cite{weisberg_quantitative_nodate}.\\

% Sección 3: Análisis metodológico escogido
% Fecha de entrega 2: 19/12/25
% Comentarios: Deben quedar claramente diferenciados el enfoque metodológico y los resultados que desde el punto de vista del alumno se asocian a la asignatura de Modelos estadísticos y los que se asocian a la asignatura de Cuantificación de riesgos, indicando como decíamos en el punto anterior, sus contribuciones fundamentales. La asociación a cada asignatura debe ser fácilmente localizable en el texto.
\section{Análisis metodológico escogido}
    El objetivo inicial es comprender la estructura del conjunto de datos y familiarizarnos con las variables disponibles, ya que esto constituye la base para cualquier análisis exploratorio. En el Cuadro 1 se presenta una descripción resumida de las variables correspondientes a una cartera de una aseguradora en Francia, que incluyen características del vehículo, del conductor, de la póliza y los montos asociados a los reclamos, entre otros.
    
    \begin{table}[H]
    \centering
    \caption{Diccionario de Variables}
    \begin{tabularx}{\textwidth}{llX}
    \toprule
    \textbf{Variable} & \textbf{Tipo\textsuperscript{(1)}} & \textbf{Descripción} \\
    \midrule
    IDpol & 2 & Número de póliza. \\
    ClaimNb & 2 & Número de siniestros. \\
    Exposure & 2 & Tiempo de vigencia y exposición al riesgo, en años. \\
    Power & 1 &Potencia del coche (en orden ascendente, de d a o). \\
    CarAge & 2 & Antigüedad del vehículo, en años. \\
    DriverAge & 2 & Edad del conductor, en años. \\
    Brand & 1 & Marca del vehículo. \\
    Gas & 1 & Tipo de combustible: Diesel o Regular. \\
    Density & 2 & Número de habitantes por km² en la ciudad del conductor. \\
    Region & 1 & Región de la póliza en Francia. \\
    ClaimAmount & 2 & Costo total del reclamo. \\
    InjuryAmount & 2 & Costo de compensación por lesiones corporales. \\
    PropertyAmount & 2 & Costo por daños materiales. \\
    \bottomrule
    \end{tabularx}
    \vspace{0.5em}
    \raggedright
    \footnotesize{\textsuperscript{(1)} 1 = variable categórica, 2 = numérica}
    \end{table}

    % A realizar entre el 01/11/25 y el 21/11/25
    \subsection{Análisis descriptivo univariado y bivariado de la base de datos}
    
    	\subsubsection{Análisis descriptivo univariado}
    	Se realiza un análisis descriptivo de las variables numéricas considerando medidas de tendencia central, dispersión y forma (\hyperref[Anexo01]{\textcolor{blue}{Cuadro 2}}). Adicionalmente, para las variables Exposure, CarAge, DriverAge y Density se aplica una técnica de segmentación mediante el método de k-means (\hyperref[Anexo02]{\textcolor{blue}{Figura 1}}). Concluido este análisis, se procede al estudio de las variables categóricas, examinando su distribución de frecuencias y representaciones gráficas para identificar patrones relevantes. Los aspectos más significativos se detallan a continuación:
    		\begin{itemize}
    			\item \textbf{Número de siniestros (ClaimNb):} Presenta valores muy bajos, con media de 0,04 y mediana cero, debido a que el 96,1 \% de las pólizas no registraron siniestros. La distribución es altamente sesgada a la derecha (asimetría 5,78) y leptocúrtica (curtosis 38,79), concentrada en cero pero con algunos valores extremos que generan gran variabilidad relativa, reflejada en una desviación estándar de 0,22 frente a una media muy pequeña.
    			\item \textbf{Tiempo de vigencia y exposición al riesgo (Exposure):} Caracterizada por una duración promedio de medio año (media 0,56; mediana 0,54) con baja variabilidad (desviación estándar 0,37), es decir, pólizas con duraciones similares. La distribución es prácticamente simétrica (asimetría -0,05) y platicúrtica (curtosis -1,57), más plana que la normal, con menor concentración en torno a la media y colas ligeras.
    			\item \textbf{Antigüedad del vehículo (CarAge):} Tiene una media de 7,5 años y una alta dispersión (desviación estándar 5,76), lo que refleja grandes diferencias entre autos nuevos y antiguos, con casos extremos de hasta 100 años. La distribución está sesgada a la derecha (asimetría 1,21), predominando vehículos relativamente nuevos, y es leptocúrtica (curtosis 8,3), concentrada cerca de la media pero con mayor probabilidad de valores extremos.
    			\item \textbf{Edad del conductor (DriverAge):} Una media de 45,3 años y una mediana cercana (44), lo que indica una distribución equilibrada. Presenta un rango amplio (18 a 99 años) y una dispersión moderada (desviación estándar 14,33), reflejando diversidad sin valores extremos desproporcionados. La distribución muestra un ligero sesgo hacia edades mayores (asimetría 0,46) y una forma cercana a la normal (curtosis -0,3), sin colas pronunciadas.
    			\item \textbf{Densidad poblacional por km2 en la ciudad del conductor (Density):} Presenta una media muy superior a la mediana (1987.33 vs 287), lo que indica una distribución fuertemente sesgada hacia valores altos. La gran variabilidad (desviación estándar 4779.6) y la asimetría positiva (4.13) reflejan diferencias significativas entre zonas de baja y alta densidad. La curtosis (17.72) confirma una distribución leptocúrtica, con concentración en valores bajos y presencia de colas largas hacia la derecha.
    			\item \textbf{Costo total del reclamo (ClaimAmount):} muestra una distribución extremadamente desbalanceada y con valores atípicos. El promedio es de 832,57 euros, pero la mediana es cero, lo que indica que más del 50 \% de las pólizas no tienen reclamos. El mínimo también es cero, mientras que el máximo alcanza los 20.368.330 euros, evidenciando la presencia de valores muy altos. La desviación estándar, de 41.847 euros, refleja una gran dispersión, y tanto la asimetría (375,9) como la curtosis (166.362,6) confirman que la distribución está fuertemente sesgada hacia la derecha, con alta concentración en valores bajos y numerosos outliers. Dado que el 96,1 \% de las pólizas no registraron reclamos, se elaboraron histogramas (\hyperref[Anexo03]{\textcolor{blue}{Figura 2}}) considerando únicamente aquellas con siniestro, para evitar que la concentración en cero oculte el comportamiento del resto:
    				\begin{itemize}
    					\item El primer histograma, que corresponde al 99 \% de las pólizas con siniestro, muestra que la mayoría de los reclamos son de bajo coste: el 96,9 \% no supera los 50.000 euros, solo el 2,4 \% está entre 50.000 y 100.000 euros y menos del 1 \% excede los 100.000 euros. El coste promedio es de 13.560 euros, lo que confirma que los siniestros de alto importe son excepcionales y que existe una fuerte concentración en valores bajos.
    					\item El histograma del 1 \% de pólizas más costosas muestra que, aunque pertenecen a la cola de la distribución, el 98,8 \% de los reclamos extremos se concentra entre 161.500 y 5.000.000 euros, con un promedio de 577.685 euros. Los valores más altos, entre 10 y 25 millones, apenas representan el 0,6 \% cada uno, confirmando que los siniestros de importe máximo son casos excepcionales con gran impacto potencial.
    				\end{itemize}
    			\item \textbf{Costo de compensación por daños corporales (InjuryAmount):} Indica una fuerte concentración en valores nulos y una cola muy prolongada con pocos reclamos de gran magnitud. La media es 615,9 euros, mientras que la mediana y el mínimo son cero, y el máximo alcanza 19.972.821 euros, evidenciando casos excepcionales. La dispersión es elevada (desviación estándar 40.867 euros) y la distribución está fuertemente sesgada a la derecha (asimetría 375,25) y leptocúrtica (curtosis 165.244,5), con alta concentración en valores bajos y presencia de outliers. Histogramas con pólizas con siniestros confirman concentración en valores bajos y una cola con pocos casos de gran coste (\hyperref[Anexo03]{\textcolor{blue}{Figura 2}}).
    			\item \textbf{Costo por daños materiales (PropertyAmount):} La distribución muestra una fuerte concentración en valores nulos y casos positivos menos extremos que en \textit{InjuryAmount}. El promedio es 216,7 euros, la mediana y el mínimo son cero, y el máximo llega a 575.508,8 euros, lo que indica reclamos significativos pero no desproporcionados. La dispersión es alta (desviación estándar: 1.554 euros) y la asimetría (129,54) y curtosis (45.682,36) confirman un sesgo marcado hacia la derecha. Para evitar que los ceros oculten el patrón real, se elaboraron dos histogramas: el primero concentra el 99 \% de los casos por debajo de 12.000 euros, destacando los rangos de 8.000-10.000 euros (26,3 \%), 4.000-6.0000 euros (23,7 \%) y  2.000-4.000 euros (22,3 \%). El segundo muestra el 1 \% restante, casi todo entre 10.400 y 100.000 euros, con algunos reclamos aislados de mayor coste. Aunque existen siniestros elevados, la mayoría se sitúa en valores moderados. Ambos gráficos se incluyen en los anexos (\hyperref[Anexo03]{\textcolor{blue}{Figura 2}}).
    		\end{itemize}
		Tras el análisis detallado de las variables numéricas, se continúa con el estudio de las variables categóricas, examinando su distribución de frecuencias y representación gráfica (\hyperref[Anexo04]{\textcolor{blue}{Figura 3}}) para identificar patrones relevantes:
			\begin{itemize}
				\item \textbf{Potencia del coche (Power):} Representa la potencia del vehículo desde la menor (d) hasta la mayor (o), cuenta con 12 categorías distintas. La categoría más frecuente es f, con 95.902 registros, lo que equivale al 23,2 \% del total. Las categorías d, e, f y g concentran la mayor parte de los datos, con porcentajes de 16,5 \% , 18,6 \% , 23,2 \%  y 22,1 \%  respectivamente, sumando en conjunto aproximadamente 80,4 \%  del total, mientras que las demás categorías, representan solo el 13,2 \% , lo que indica que la mayoría de los vehículos se sitúan en rangos de potencia media.
				\item \textbf{Marca del vehículo (Brand):} Cuenta con 7 categorías distintas, la más frecuente es “Renault, Nissan or Citroen”, con 218.591 registros, lo que equivale aproximadamente al 52,8 \% del total, mostrando una clara predominancia de estas marcas en el conjunto de datos. Le siguen Japanese (except Nissan) or Korean con 79228 registros (19,1 \%), Opel, General Motors or Ford con 37477 (9,0 \%) y Volkswagen, Audi, Skoda or Seat con 32707 (7,9 \%), mientras que el resto de marcas representan solo el 11,1 \%. Esta distribución indica que más de la mitad de los vehículos pertenecen a marcas del grupo Renault-Nissan-Citroën, evidenciando una concentración significativa en este segmento.
				\item \textbf{Tipo de combustible (Gas):} Presenta únicamente 2 categorías, regular y diesel. La categoría más frecuente es regular, con 207.610 registros, lo que representa aproximadamente el 50,2 \% del total, mientras que Diesel concentra el 49,8 \% restante. Esta distribución muestra un equilibrio casi perfecto entre ambos tipos de combustible.
				\item \textbf{Región de la póliza (Region):} Cuenta con 10 categorías diferentes, la más frecuente es Centre, con 160.814 registros, lo que representa aproximadamente el 38,8 \% del total, seguida por Île-de-France con 16,9 \%, Bretagne con 10,2 \%, Pays-de-la-Loire con 9,4 \%, Aquitaine con 7,6 \% y Nord-Pas-de-Calais con 6,6 \%, mientras que el resto de regiones suman el 10,5 \%. Esta distribución evidencia una fuerte concentración en la región Centre, que por sí sola agrupa más de un tercio de los registros, mientras que las demás regiones presentan proporciones significativamente menores.
			\end{itemize}

		\subsubsection{Análisis descriptivo bivariado}
		
		El análisis de correlación entre las variables numéricas muestra que la mayoría de las relaciones son débiles o prácticamente nulas, lo que indica baja dependencia lineal entre ellas (Pearson). Destacan correlaciones muy altas, como era de esperarse, entre\textit{ ClaimAmount} e \textit{InjuryAmount} ($\approx 99\%$), así como entre \textit{ClaimAmount} y \textit{PropertyAmount} ($\approx 64 \%$). También se observa una correlación considerable entre \textit{ClaimNb} y \textit{PropertyAmount} ($\approx 66 \%$), indicando que un mayor número de reclamaciones tiende a relacionarse con mayores daños materiales. El resto de las variables, como \textit{CarAge}, \textit{DriverAge}, \textit{Density} y \textit{Exposure}, presentan correlaciones muy bajas con las variables de coste, lo que sugiere que no influyen significativamente en el valor de los siniestros. En general, el patrón confirma que las variables monetarias están fuertemente interrelacionadas, mientras que las características del vehículo y del conductor tienen poca relación lineal con los montos reclamados. En el Anexo se incluye la matriz de correlaciones (\hyperref[Anexo05]{\textcolor{blue}{Cuadro 3}}), así como un mapa de calor (\hyperref[Anexo06]{\textcolor{blue}{Figura 4}}). \\
		
		Se analiza la relación entre variables categóricas y dos indicadores clave: el número de reclamos (\textit{ClaimNb}) y el coste de daños corporales (\textit{InjuryAmount}).\\
		
		Los vehículos de menor potencia concentran más siniestros, alcanzando hasta 3 o 4 reclamos por póliza, mientras que los de mayor potencia no superan los dos, lo que indica que la potencia no incrementa la frecuencia de siniestros (\hyperref[Anexo07]{\textcolor{blue}{Figura 5}}). En cuanto al coste de lesiones, la mayoría de los valores se sitúan cerca de cero, reflejando montos bajos en la mayoría de los casos. Sin embargo, se observan outliers en categorías como f y n, con cifras cercanas a 20.000.000 euros. En general, no se aprecia una relación clara entre la potencia y el importe de las lesiones corporales (\hyperref[Anexo08]{\textcolor{blue}{Figura 6}}).\\
		
		En cuanto a la marca del vehículo, se observa que la frecuencia de siniestros se mantiene en niveles bajos para la mayoría de las categorías, con valores que oscilan entre 1 y 2 reclamos por póliza. Sin embargo, existen casos puntuales que alcanzan hasta 4 reclamos, especialmente en marcas japonesas (excepto Nissan) o coreanas, así como en Opel, General Motors o Ford (\hyperref[Anexo09]{\textcolor{blue}{Figura 7}}). En cuanto al coste por daños corporales, la mayoría de los valores se sitúan cerca de cero, reflejando importes reducidos en la mayoría de los casos. No obstante, destacan dos casos extremos que corresponden a pólizas de vehículos de marca Renault, Nissan o Citroën, con cifras que alcanzan aproximadamente los 20.000.000 euros (\hyperref[Anexo10]{\textcolor{blue}{Figura 8}}).\\
		
		Respecto al tipo de combustible, se observa que la frecuencia de siniestros se mantiene en niveles similares tanto para vehículos diésel como para los de gasolina regular, con valores que oscilan entre 1 y 2 reclamos por póliza. No obstante, se presentan casos puntuales que alcanzan hasta 3 o 4 reclamos en ambas categorías, lo que indica que el tipo de combustible no incrementa la frecuencia de siniestros (\hyperref[Anexo11]{\textcolor{blue}{Figura 9}}). En relación con el coste de lesiones corporales, la mayoría de los valores se sitúan cerca de cero, reflejando importes bajos en la mayoría de los casos. Sin embargo, se identifican outliers significativos en vehículos de gasolina regular (\hyperref[Anexo12]{\textcolor{blue}{Figura 10}}).\\
		
		La frecuencia de siniestros se mantiene en niveles bajos para la mayoría de las regiones, con valores que oscilan entre 1 y 2 reclamos por póliza, aunque se observan casos puntuales que alcanzan hasta 3 o 4 reclamos en Centre, Ile-de-France y Nord-Pas-de-Calais, mientras que Basse-Normandie, Haute-Normandie y Limousin destacan por presentar un menor número de siniestros (\hyperref[Anexo13]{\textcolor{blue}{Figura 11}}. En lo referente al coste de lesiones corporales, la mayoría de los valores se sitúan cerca de cero, reflejando importes reducidos en la mayoría de los casos; sin embargo, se identifican outliers significativos en la región Centre, donde se registran los costes más extremos (\hyperref[Anexo14]{\textcolor{blue}{Figura 12}}).
		
    % A realizar entre el 22/11/25 y el 19/12/25
    \subsection{Modelización seleccionada y objetivos a alcanzar}
        \subsubsection{Número de siniestros (Modelización estadística)}
        	A continuación, desarrollaremos los distintos modelos estudiados a lo largo de la asignatura, indicando en cada caso las razones por las que no resultan aplicables cuando corresponda.
        	\begin{enumerate}
        		\item \textbf{Modelo de regresión lineal clásico:} No resulta adecuado cuando la variable dependiente es discreta, ya que este modelo parte del supuesto de que dicha variable es continua y sigue una distribución normal, condición que no se cumple en este caso.
        		\item \textbf{Modelo Poisson:} método estadístico donde los datos son números enteros no negativos, que permite identificar qué variables explicativas influyen en la frecuencia de un evento.
        		\item \textbf{Modelos lineales generalizados (GLM):} Existen diversos modelos que constituyen casos particulares. En este trabajo nos centraremos en aquellos que se aplican a variables dependientes de elección discreta.
        			\begin{enumerate}
        				\item \textbf{Elección binaria - logit y probit:} Describe situaciones en las que la variable dependiente del modelo econométrico es discreta y toma únicamente dos valores posibles. En nuestro caso, utilizaremos variables ficticias binarias que adoptan los valores 0 y 1, donde 1 indica que el individuo ha tenido al menos un siniestro y 0 indica que no ha tenido siniestros. En estos modelos, se supone una relación no lineal entre las variables. El término de error sigue una distribución logística (modelo logit) o normal (probit).
        				\item \textbf{Elección multinomial:} No lo utilizamos porque este tipo de modelo está pensado para variables cualitativas con categorías sin orden, como elegir entre colores o marcas. En nuestro caso, el número de siniestros (0, 1, 2, …) tiene un orden natural y representa cantidades, no categorías independientes. Si tratáramos cada número como una categoría separada, el modelo ignoraría esa relación.
        				\item \textbf{Elección multinomial ordenada:} Se utiliza cuando la variable dependiente tiene categorías con orden, pero no es estrictamente numérica. Por ejemplo, medir el nivel de satisfacción: bajo, medio y alto. Aquí hay un orden, pero no sabemos si la distancia entre 'bajo' y 'medio' es igual a la de 'medio' y 'alto'. En cambio, el número de siniestros es una variable de conteo, donde las diferencias sí son cuantitativas y significativas.
        				\item \textbf{Elección multinomial anidada:} Está pensado para decisiones jerárquicas entre alternativas cualitativas, no para cantidades. Usarlo para número de siniestros sería conceptualmente incorrecto porque no hay un proceso de elección jerárquico.
        			\end{enumerate}
   				\item \textbf{Cluster jerárquico aglomerativo:} Pendiente
        		\item \textbf{Cluster jerárquico divisivo:} Víctor Alonso
        		\item \textbf{Cluster no jerárquico - k means:} Es un método de agrupamiento que busca dividir un conjunto de datos en 'k' grupos (clusters) basados en similitud. Es un algoritmo iterativo que elige 'k' centros (centroides) iniciales y asigna cada observación al centro más cercano. Luego recalcula cada centro como la media de los puntos en su grupo y repite este proceso hasta que las asignaciones no cambian. El objetivo es que los grupos sean lo más parecidos posibles entre sí y diferentes de los otros grupos.\\
        		
        		Para aplicar el algoritmo k-means, se seleccionaron las variables \textit{Exposure}, \textit{CarAge}, \textit{DriverAge}, \textit{Density} y \textit{Power} (esta última convertida previamente a formato numérico). Todas fueron normalizadas mediante estandarización ($\frac{x-media}{desv. estandar}$). \\
        		
        		Para determinar el número óptimo de clusters (k), utilizamos el método del codo, que analiza la relación entre el número de clusters y la suma total de cuadrados dentro de los grupos (WSS). El punto donde la curva deja de disminuir de forma pronunciada y comienza a aplanarse se denomina 'codo'. En nuestro caso, para las variables analizadas, el número óptimo de clusters es 4 (\hyperref[Anexo15]{\textcolor{blue}{Figura 13}}). A continuación, se presentan los principales comentarios que resumen los resultados detallados en el Cuadro 4 del anexo (\hyperref[Anexo16]{\textcolor{blue}{Cuadro 4}}):
        		\begin{itemize}
        			\item \textbf{Exposure:} Clúster 4 tiene la mayor vigencia (1 año en promedio, rango de 0,8 a 2 años), representa el 35 \%. Clúster 1 la más corta (0,1 años en promedio, rango hasta 0,3 años), aporta el 31 \%. Ambos suman 66 \% del total.
        			\item \textbf{CarAge:} Clúster 1 concentra vehículos nuevos (2,7 años en promedio, rango de 0 hasta 6 años) con 49 \%. Clúster 2, antigüedad intermedia (9,3 años en promedio, rango 7–13 años), aporta 35 \%. Ambos simbolizan el 84 \%
        			\item \textbf{DriverAge:} Clúster 2 reúne edades medias (40,9 años en promedio, rango de 35–47) con 32 \%. Clúster 3, conductores mayores (54 años en promedio, rango 48–62 años), representa 29 \%.
        			\item \textbf{Density:} Clúster 1 con zonas poco pobladas (398,9 km2, rango hasta 2.317), constituyendo el 80\% del total.
        			\item \textbf{Power:} Clúster 2 concentra potencia media-baja (3,7 en promedio, rango de 3–5) con el 50 \%. Clúster 1, la más baja (1,5 en promedio, rango hasta 3), aporta 35 \%.
        		\end{itemize}
   
           		\item \textbf{Análisis PCA:} Pendiente
        	\end{enumerate}
        	
        \subsubsection{Costo del siniestro (Cuantificación de riesgos)}

% Sección 4: Resultados
% Fecha de entrega: 19/12/25
\section{Resultados}
    Por desarrollar.

% Sección 5: Informe ejecutivo
% Fecha de entrega: 10/01/26
\section{Informe ejecutivo}
    Por desarrollar.

% Sección 6: Anexos
% Fecha de entrega: 10/01/26
\section{Anexos}

% Análisis descriptivo univariado (Variables discretas y contínuas): Tabla estadísticos
	\label{Anexo01}
	\begin{table}[H]
		\centering
		\caption{Análisis descriptivo univariado - variables numéricas}
		\begin{tabularx}{\textwidth}{lrrrrr}
			\toprule
			\textbf{Concepto} & \textbf{Claim Nb} & \textbf{Exposure} & \textbf{CarAge} & \textbf{DriverAge} & \textbf{Density} \\
			\midrule
			Media & 0,04 & 0,56 & 7,53 & 45,32 & 1.987,33 \\
			Mediana & 0 & 0,54 & 7 & 44 & 287 \\
			Mínimo & 0 & 0 & 0 & 18 & 2 \\
			Máximo & 4 & 1,99 & 100 & 99 & 27000 \\
			Desv. Estándar & 0,22 & 0,37 & 5,76 & 14,33 & 4.779,6 \\
			Asimetría & 5,78 & -0,05 & 1,21 & 0,46 & 4,13 \\
			Curtosis & 38,79 & -1,57 & 8,3 & -0,3 & 17,72 \\
			\bottomrule
		\end{tabularx}
	\end{table}
	\begin{table}[H]\ContinuedFloat
		\centering
		\caption{Análisis descriptivo univariado - variables numéricas}
		\begin{tabularx}{\textwidth}{lrrr}
			\toprule
			\textbf{Concepto} & \textbf{Claim Amount} & \textbf{Injury Amount} & \textbf{Property Amount} \\
			\midrule
			Media & 832,57 & 615,9 & 216,67 \\
			Mediana & 0 & 0 & 0 \\
			Mínimo & 0 & 0 & 0 \\
			Máximo & 20.368.330 & 19.792.821 & 575.508,8 \\
			Desv. Estándar & 41.847 & 40.867,4 & 1.554,24 \\
			Asimetría & 375,9 & 375,25 & 129,54 \\
			Curtosis & 166.362,6 & 165.244,5 & 45.682,36 \\
			\bottomrule
		\end{tabularx}
	\end{table}

% Análisis descriptivo univariado (Variables numéricas)
	\label{Anexo02}
	\begin{figure}[htbp]
		\centering
		\caption{Análisis descriptivo univariado}
		% Primera fila
		\begin{subfigure}[b]{0.4\textwidth}
			\includegraphics[width=\textwidth]{Gráfico01.png}
		\end{subfigure}
		\hfill
		\begin{subfigure}[b]{0.4\textwidth}
			\includegraphics[width=\textwidth]{Gráfico02.png}
		\end{subfigure}
		% Segunda fila
		\vspace{0.5cm} % Espacio entre filas
		\begin{subfigure}[b]{0.4\textwidth}
			\includegraphics[width=\textwidth]{Gráfico03.png}
		\end{subfigure}
		\hfill
		\begin{subfigure}[b]{0.4\textwidth}
			\includegraphics[width=\textwidth]{Gráfico04.png}
		\end{subfigure}
		% Tercera fila
		\begin{subfigure}[b]{0.4\textwidth}
			\includegraphics[width=\textwidth]{Gráfico05.png}
		\end{subfigure}
		\hfill
	\end{figure}

% Análisis descriptivo univariado (Variables numéricas): Histogramas
\label{Anexo03}
\begin{figure}[htbp]
	\centering
	\caption{Análisis descriptivo univariado (Variables numéricas) - Histogramas}
	% Primera fila
	\begin{subfigure}[b]{0.4\textwidth}
		\includegraphics[width=\textwidth]{Gráfico06.png}
	\end{subfigure}
	\hfill
	\begin{subfigure}[b]{0.4\textwidth}
		\includegraphics[width=\textwidth]{Gráfico07.png}
	\end{subfigure}
	% Segunda fila
	\vspace{0.5cm} % Espacio entre filas
	\begin{subfigure}[b]{0.4\textwidth}
		\includegraphics[width=\textwidth]{Gráfico08.png}
	\end{subfigure}
	\hfill
	\begin{subfigure}[b]{0.4\textwidth}
		\includegraphics[width=\textwidth]{Gráfico09.png}
	\end{subfigure}
	% Tercera fila
	\begin{subfigure}[b]{0.4\textwidth}
		\includegraphics[width=\textwidth]{Gráfico10.png}
	\end{subfigure}
	\hfill
	\begin{subfigure}[b]{0.4\textwidth}
		\includegraphics[width=\textwidth]{Gráfico11.png}
	\end{subfigure}
\end{figure}

% Análisis descriptivo univariado (Variables categóricas): Gráfico de pastel
\label{Anexo04}
\begin{figure}[htbp]
	\centering
	\caption{Análisis descriptivo univariado (Variables categóricas) - Gráfico de pastel}
	% Primera fila
	\begin{subfigure}[b]{0.4\textwidth}
		\includegraphics[width=\textwidth]{Gráfico12.png}
	\end{subfigure}
	\hfill
	\begin{subfigure}[b]{0.4\textwidth}
		\includegraphics[width=\textwidth]{Gráfico13.png}
	\end{subfigure}
	% Segunda fila
	\begin{subfigure}[b]{0.4\textwidth}
		\includegraphics[width=\textwidth]{Gráfico14.png}
	\end{subfigure}
	\hfill
	\begin{subfigure}[b]{0.4\textwidth}
		\includegraphics[width=\textwidth]{Gráfico15.png}
	\end{subfigure}
\end{figure}

% Análisis descriptivo bivariado (Variables discretas y contínuas): Matriz de correlación de Pearson
\label{Anexo05}
\begin{table}[H]
	\centering
	\caption{Matriz de correlación de Pearson}
	\begin{tabularx}{\textwidth}{lrrrrrr}
		\toprule
		\textbf{Variable} & \textbf{PolicyID} & \textbf{ClaimNb} & \textbf{Exposure} & \textbf{CarAge} & \textbf{DriverAge} & \textbf{Density} \\
		\midrule
		PolicyID & 1 & - & - & - & - & - \\
		ClaimNb & -0,0327 & 1 & - & - & - & - \\
		Exposure & -0.1324 & 0,0761 & 1 & - & - & - \\
		CarAge & -0.0789 & 0,0025 & 0,1399 & 1 & - & - \\
		DriverAge & 0,0487 & -0,0075 & 0,1943 & -0,0465 & 1 & -  \\
		Density & 0,1022 & 0,0089 & -0,1121 & -0,1423 & -0,0016 & 1 \\
		ClaimAmount & -0,0054 & 0,0977 & 0,0022 & 0,0016 & -0,0045 & -0,0013 \\
		InjuryAmount & -0,0041 & 0,0749 & 0,0002 & 0,0014 & -0,0045 & -0,0014 \\
		PropertyAmount & -0,0389 & 0,6620 & 0,0559 & 0,0062 & -0,0038 & 0,0012 \\
		\bottomrule
	\end{tabularx}
\end{table}
\begin{table}[H]\ContinuedFloat
	\centering
	\caption{Matriz de correlación de Pearson}
	\begin{tabularx}{\textwidth}{lrrr}
		\toprule
		\textbf{Concepto} & \textbf{Claim Amount} & \textbf{Injury Amount} & \textbf{Property Amount} \\
		\midrule
		PolicyID & - & - & - \\
		ClaimNb & - & - & - \\
		Exposure & - & - & - \\
		CarAge & - & - & - \\
		DriverAge & - & - & - \\
		Density & - & - & - \\
		ClaimAmount & 1 & - & - \\
		InjuryAmount & 0,9996 & 1 & - \\
		PropertyAmount & 0,6417 & 0,6190 & 1 \\
		\bottomrule
	\end{tabularx}
\end{table}

% Análisis descriptivo bivariado (Variables discretas y contínuas): Matriz de correlación de Pearson (Mapa de calor)
\label{Anexo06}
\begin{figure}[htbp]
	\centering
	\caption{Matriz de correlación de Pearson - Mapa de calor}
	\begin{subfigure}[b]{0.9\textwidth}
		\includegraphics[width=\textwidth]{Gráfico16.png}
	\end{subfigure}
\end{figure}

% Análisis descriptivo bivariado (Variables categóricas): Jitter plot - Power vs ClaimNb
\label{Anexo07}
\begin{figure}[htbp]
	\centering
	\caption{Jitter plot - Power vs ClaimNb}
	% Primera fila
	\begin{subfigure}[b]{0.45\textwidth}
		\includegraphics[width=\textwidth]{Gráfico17.png}
	\end{subfigure}
	\hfill
	\begin{subfigure}[b]{0.45\textwidth}
		\includegraphics[width=\textwidth]{Gráfico18.png}
	\end{subfigure}
	% Segunda fila
	\begin{subfigure}[b]{0.45\textwidth}
		\includegraphics[width=\textwidth]{Gráfico19.png}
	\end{subfigure}
	\hfill
	\begin{subfigure}[b]{0.45\textwidth}
		\includegraphics[width=\textwidth]{Gráfico20.png}
	\end{subfigure}
\end{figure}

% Análisis descriptivo bivariado (Variables categóricas): Jitter plot - Power vs InjuryAmount
\label{Anexo08}
\begin{figure}[htbp]
	\centering
	\caption{Jitter Plot - Power vs InjuryAmount}
	% Primera fila
	\begin{subfigure}[b]{0.45\textwidth}
		\includegraphics[width=\textwidth]{Gráfico21.png}
	\end{subfigure}
	\hfill
	\begin{subfigure}[b]{0.45\textwidth}
		\includegraphics[width=\textwidth]{Gráfico22.png}
	\end{subfigure}
	% Segunda fila
	\begin{subfigure}[b]{0.45\textwidth}
		\includegraphics[width=\textwidth]{Gráfico23.png}
	\end{subfigure}
	\hfill
	\begin{subfigure}[b]{0.45\textwidth}
		\includegraphics[width=\textwidth]{Gráfico24.png}
	\end{subfigure}
\end{figure}

% Análisis descriptivo bivariado (Variables categóricas): Jitter plot - Brand vs ClaimNb
\label{Anexo09}
\begin{figure}[htbp]
	\centering
	\caption{Jitter plot - Brand vs ClaimNb}
	% Primera fila
	\begin{subfigure}[b]{0.45\textwidth}
		\includegraphics[width=\textwidth]{Gráfico25.png}
	\end{subfigure}
	\hfill
	\begin{subfigure}[b]{0.45\textwidth}
		\includegraphics[width=\textwidth]{Gráfico26.png}
	\end{subfigure}
	% Segunda fila
	\begin{subfigure}[b]{0.45\textwidth}
		\includegraphics[width=\textwidth]{Gráfico27.png}
	\end{subfigure}
\end{figure}

% Análisis descriptivo bivariado (Variables categóricas): Jitter plot - Brand vs InjuryAmount
\label{Anexo10}
\begin{figure}[htbp]
	\centering
	\caption{Jitter Plot - Brand vs InjuryAmount}
	% Primera fila
	\begin{subfigure}[b]{0.45\textwidth}
		\includegraphics[width=\textwidth]{Gráfico28.png}
	\end{subfigure}
	\hfill
	\begin{subfigure}[b]{0.45\textwidth}
		\includegraphics[width=\textwidth]{Gráfico29.png}
	\end{subfigure}
	% Segunda fila
	\begin{subfigure}[b]{0.45\textwidth}
		\includegraphics[width=\textwidth]{Gráfico30.png}
	\end{subfigure}
\end{figure}

% Análisis descriptivo bivariado (Variables categóricas): Jitter plot - Gas vs ClaimNb
\label{Anexo11}
\begin{figure}[htbp]
	\centering
	\caption{Jitter plot - Gas vs ClaimNb}
	% Primera fila
	\begin{subfigure}[b]{0.45\textwidth}
		\includegraphics[width=\textwidth]{Gráfico31.png}
	\end{subfigure}
\end{figure}

% Análisis descriptivo bivariado (Variables categóricas): Jitter plot - Gas vs InjuryAmount
\label{Anexo12}
\begin{figure}[htbp]
	\centering
	\caption{Jitter Plot - Gas vs InjuryAmount}
	% Primera fila
	\begin{subfigure}[b]{0.45\textwidth}
		\includegraphics[width=\textwidth]{Gráfico32.png}
	\end{subfigure}
\end{figure}

% Análisis descriptivo bivariado (Variables categóricas): Jitter plot - Region vs ClaimNb
\label{Anexo13}
\begin{figure}[htbp]
	\centering
	\caption{Jitter plot - Region vs ClaimNb}
	% Primera fila
	\begin{subfigure}[b]{0.45\textwidth}
		\includegraphics[width=\textwidth]{Gráfico33.png}
	\end{subfigure}
	\hfill
	\begin{subfigure}[b]{0.45\textwidth}
		\includegraphics[width=\textwidth]{Gráfico34.png}
	\end{subfigure}
	% Segunda fila
	\begin{subfigure}[b]{0.45\textwidth}
		\includegraphics[width=\textwidth]{Gráfico35.png}
	\end{subfigure}
	\hfill
	\begin{subfigure}[b]{0.45\textwidth}
		\includegraphics[width=\textwidth]{Gráfico36.png}
	\end{subfigure}
\end{figure}

% Análisis descriptivo bivariado (Variables categóricas): Jitter plot - Region vs InjuryAmount
\label{Anexo14}
\begin{figure}[htbp]
	\centering
	\caption{Jitter Plot - Region vs InjuryAmount}
	% Primera fila
	\begin{subfigure}[b]{0.45\textwidth}
		\includegraphics[width=\textwidth]{Gráfico37.png}
	\end{subfigure}
	\hfill
	\begin{subfigure}[b]{0.45\textwidth}
		\includegraphics[width=\textwidth]{Gráfico38.png}
	\end{subfigure}
	% Segunda fila
	\begin{subfigure}[b]{0.45\textwidth}
		\includegraphics[width=\textwidth]{Gráfico39.png}
	\end{subfigure}
	\hfill
	\begin{subfigure}[b]{0.45\textwidth}
		\includegraphics[width=\textwidth]{Gráfico40.png}
	\end{subfigure}
\end{figure}

% Modelización seleccionada y objetivos a alcanzar: Número de siniestros - Cluster no jerárquico - k means
\label{Anexo15}
\begin{figure}[htbp]
	\centering
	\caption{Cluster no jerárquico - k means}
	% Primera fila
	\begin{subfigure}[b]{0.45\textwidth}
		\includegraphics[width=\textwidth]{Gráfico41.png}
	\end{subfigure}
	\hfill
	\begin{subfigure}[b]{0.45\textwidth}
		\includegraphics[width=\textwidth]{Gráfico42.png}
	\end{subfigure}
	% Segunda fila
	\begin{subfigure}[b]{0.45\textwidth}
		\includegraphics[width=\textwidth]{Gráfico43.png}
	\end{subfigure}
	\hfill
	\begin{subfigure}[b]{0.45\textwidth}
		\includegraphics[width=\textwidth]{Gráfico44.png}
	\end{subfigure}
	% Tercera fila
	\begin{subfigure}[b]{0.45\textwidth}
		\includegraphics[width=\textwidth]{Gráfico45.png}
	\end{subfigure}
\end{figure}

% Modelización seleccionada y objetivos a alcanzar: Número de siniestros - Cluster no jerárquico - k means
\label{Anexo16}
\begin{table}[H]
	\centering
	\caption{Clúster no jerárquico - k means}
	\begin{tabularx}{\textwidth}{llrrrrr}
		\toprule
		\textbf{Concepto} & \textbf{Cluster} & \textbf{Exposure} & \textbf{CarAge} & \textbf{DriverAge} & \textbf{Density} & \textbf{Power} \\
		\midrule
		\multirow{4}{*}{Centroide estand.} 
		& 1 & -1,2 & -0,8 & -1,2 & -0,3 & -0,9\\
		& 2 & -0,4 & 0,4 & -0,3 & 0,5 & 0,1\\
		& 3 & 0,4 & 1,5 & 0,6 & 2,3 & 1,7\\
		& 4 & 1,2 & 4,2 & 1,8 & 5,2 & 3,3\\
		\midrule
		\multirow{4}{*}{Centroide} 
		& 1 & 0,1 & 2,7 & 28,5 & 398,9 & 1,5\\
		& 2 & 0,4 & 9,9 & 40,9 & 4.257,8 & 3,7\\
		& 3 & 0,7 & 16,2 & 53,9 & 12.810,2 & 6,8\\
		& 4 & 1,0 & 32,0 & 71,1 & 26.692,8 & 10,0\\
		\midrule
		\multirow{4}{*}{Mínimo} 
		& 1 & 0,0 & 0 & 18 & 2 & 1\\
		& 2 & 0,3 & 7 & 35 & 2.348 & 3\\
		& 3 & 0,6 & 14 & 48 & 8.561 & 6\\
		& 4 & 0,8 & 25 & 63 & 20.000 & 9\\
		\midrule
		\multirow{4}{*}{Máximo} 
		& 1 & 0,3 & 6 & 34 & 2.317 & 3\\
		& 2 & 0,6 & 13 & 47 & 8.346 & 5\\
		& 3 & 0,8 & 24 & 62 & 18.229 & 8\\
		& 4 & 2,0 & 100 & 99 & 27.000 & 12\\
		\midrule
		\multirow{4}{*}{Tamaño} 
		& 1 & 128.390 & 201.485 & 108.167 & 329.523 & 145.339\\
		& 2 & 83.237 & 144.258 & 132.408 & 64.073 & 214.001\\
		& 3 & 58.021 & 65.981 & 120.016 & 9.015 & 45.274\\
		& 4 & 144.312 & 2.236 & 53.369 & 11.349 & 9.346\\
		\midrule
		\bottomrule
	\end{tabularx}
\end{table} 


% Referencias
\newpage
\addcontentsline{toc}{section}{Referencias}
\bibliographystyle{apalike}
\bibliography{FC01_G03_Referencias.bib}

\end{document}